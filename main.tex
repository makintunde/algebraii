\documentclass{article}
\usepackage{amsmath}
\usepackage[utf8]{inputenc}
\usepackage{amssymb}
\usepackage{array}
\usepackage{amsthm}
\usepackage{parskip}

\newtheorem{theorem}{Theorem}[section]

\theoremstyle{definition}
\newtheorem{definition}{Definition}[section]

\newtheorem{lemma}[theorem]{Lemma}
\newtheorem{proposition}[theorem]{Proposition}
\newtheorem{corollary}[theorem]{Corollary}
\newtheorem{remark}{Remark}

\newcommand{\dtn}{\Delta_n}
\DeclareMathOperator{\sgn}{sgn}


\title{Algebra II}
\author{John R. Britnell - 655\\
        Problems class: 12:00pm\\
        Office hour: Mon 2:00pm}
\date{October 2015}

\begin{document}

\maketitle

\section{Groups}
A group is a set $G$ with a binary operation $*$ such that 
\begin{itemize}
\item Associativity: $(x * y)*z = x * (y * z)$ for all $x,y,z \in G$.
\item Identity: There is $e \in G$ such that $e * x = x * e$ for all $x \in G$.
\item Inverses: For all $x \in G$ there exists $y \in G$ such that $x * y = yz * y = e$
\end{itemize}

\subsection{Some groups}
\begin{enumerate}
\item
\begin{itemize}
\item $\mathbb{Z}_n$, integers modulo $n$ under $+$.
\item $\mathbb{Z}$, integers under $+$.
\end{itemize}
These are \emph{cyclic groups}.
\item
\begin{itemize}
\item $D_{2n}$, $n \geq 3$ dihedral group (of rotations and reflections) of a regular $n$-gon.
\item We also have $D_\infty$ the \emph{infinite dihedral group}.

Take the ``polygon''.

What are the ``rotations''? These are shifts: move each vertex some number $k$ of places to the right. (if $k$ is -ve we shift left instead). 

``Reflections'' really are reflections - through a vertical axis, either through a vertex or the midpoint of an ``edge''.

The ``rotations'' and ``reflections'' form a group under composition. This is $D_\infty$. The subgroup of rotations is an infinite cyclic group, generated by $R_1$

\end{itemize}
\item Symmetric groups $S_n$. Permutations of $\{1, \cdots, n\}$ under composition. $|S_n| = n!$
More generally if $\Omega$ is a set then $Sym(\Omega)$, the symmetric group on $\Omega$ is the group of all permutations of $\Omega$.

\item Let $F$ be a field. Then $GL_n(F)$, the \emph{General Linear Group of degree $n$ over $F$}, is the set of \emph{invertible} $n \times n$ matrices with entries from $F$. It is a group under matrix multiplication. 
\end{enumerate}

\section{Motivating Example}

\begin{table}[h]
\centering
\begin{tabular}{l|llllll}
$s_3$  & e     & (123) & (132) & (12)  & (13)  & (23)  \\
\hline
e     & e     & (123) & (132) & (12)  & (13)  & (23)   \\
(123) & (123) & (132) & e     & (13)  & (23)  & (12)   \\
(132) & (132) & e     & (123) & (23)  & (12)  & (13)   \\
(12)  & (12)  & (23)  & (13)  & e     & (132) & (123)  \\
(13)  & (13)  & (12)  & (23)  & (123) & e     & (132)  \\
(23)  & (23)  & (13)  & (12)  & (132) & (123) & e      \\
\end{tabular}
\end{table}

These tables coincide if we make the equivalence:
$$ 
\begin{matrix}
  e & \leftrightarrow & e \\
  R & \leftrightarrow & (123) \\
  R^{-1} & \leftrightarrow & (132) \\
  U & \leftrightarrow & (12) \\
  V & \leftrightarrow & (13) \\
  W & \leftrightarrow & (23) \\
\end{matrix}
$$


%\begin{equation*}
%e &= e \\
%R &= (123) \\
%R^{-1} &= (132) \\
%U &= (12) \\
%V &= (13) \\
%W &= (23)
%\end{equation*}

From an algebraic point of view these two groups are ``the same''. We say they are \textit{isomorphic}.

Here's a formal definition.\\


\begin{definition}
Let $G$ and $H$ be groups, and let $f : G \rightarrow H$ be a function. We say that $f$ is an \emph{isomorphism} if:
\begin{enumerate}
\item $f$ is a bijection.
\item $f(g_1 * g_2)=f(g_1) * f(g_2) $ for all $g_1, g_2 \in G$.

Note that in $f(g_1 * g_2)$ the multiplication is happening in $G$, but in $f(g_1) * f(g_2)$ the multiplication is in $H$. If condition (2) is satisfied, we say that f \emph{respects} $*$.

$G$ and $H$ are \emph{isomorphic} if an isomorphism $G \rightarrow H$ exists.
\end{enumerate}
\end{definition}

\begin{itemize}
  \item 
$G \equiv H$ if $G$ and $H$ are isomorphic and $f$ is the isomorphism between them
  \item
$G \equiv H$ if $G$ and $H$ are isomorhpic via some isomorphism (particular isomorphism not mentioned)
  \item 
$G \not\equiv H$ if $G$ and $H$ are not isomorphic.
\end{itemize}

\emph{Examples}: Determine which pairs are isomorphic:

\begin{itemize}
  \item $S_2=\{e, (1, 2)\}$
  \item $\mathbb{Z}_2=\{0,1\}$ operations: + mod 2
  \item $C_2=\{1,-1\}$ operations: multiplication
\end{itemize}
$C_n$ represents $n^{th}$ roots of unity.

Look at Group Tables:

\begin{table}[h]
  \centering
\label{my-label}
\begin{tabular}{l|ll}
$s_2$  & e     & (1,2) \\
\hline
e     & e     & (1,2) \\
(1,2) & (1,2) & e    
\end{tabular}
\end{table}

\begin{table}[h]
  \centering
\label{my-label}
\begin{tabular}{l|ll}
  $\mathbb{Z}_2$  & 0  & 1\\
\hline
0     & 0     & 1 \\
1 & 1 & 0    
\end{tabular}
\end{table}

\begin{table}[h]
  \centering
\label{my-label}
\begin{tabular}{l|ll}
  $C_2$  & 1  & -1\\
\hline
1     & 1  & -1 \\
-1    & -1 & 1    
\end{tabular}
\end{table}
All groups are isomorphic, which can be shown by `relabelling': 

\begin{table}[h]
  \centering
\label{my-label}
\begin{tabular}{lllll}
  $S_2$  &  & $\mathbb{Z}_2$ & & $C_2$\\
\hline
e     & $\leftrightarrow$ & 0 & $\leftrightarrow$ & 1 \\
(1,2) & $\leftrightarrow$ & 1 & $\leftrightarrow$ & -1    
\end{tabular}
\end{table}

Are $\mathbb{Z}_3=\{0,1,2\}$ and $C_3=\{1,w,w^2\}$, where $w=e^{\frac{2\pi i}{3}}$ isomorphic?

\underline{Yes}, an isomorphism is:
\begin{equation}
  f : 
       \begin{matrix}
        0 & \mapsto & 1 \\
        1 & \mapsto & w \\
        2 & \mapsto & w^2 \\
        \mathbb{Z}_3 & \mapsto & C_3
      \end{matrix} 
\end{equation}

Another:
\begin{equation}
  \hat{f} :
       \begin{matrix}
        0 & \mapsto & 1 \\
        1 & \mapsto & w^2 \\
        2 & \mapsto & w \\
      \end{matrix} 
\end{equation}

\begin{remark}
\begin{enumerate}
  \item Let $G$ be finite and let $G \equiv H$ then $|G|=|H|$ i.e. sets are the same size, clearly since isomorphism is a bijection.
  \item Let $G$ have identiy $e_G$ and $H$ have identity $e_H$. Suppose $G \equiv H$. Then, $f(e_G)=e_H$.
  \item `$\equiv$' is an equivalence relation on groups.
   We have symmetry, hence order does not matter for isomorphism:
    \begin{itemize}
      \item $G \equiv G$
      \item $G \equiv H \iff H \equiv G$
      \item $G \equiv H, H \equiv H \implies G \equiv K \,, \forall \text{ groups } G,H,K$
    \end{itemize}
\end{enumerate}
\end{remark}

\underline{Example:} Which of these pairs of groups isomorphic?
\begin{itemize}
  \item 
$G_1 = c_4$ = \{1,-1,i,-i\}
  \item
$G_2 = $ group of \emph{symmetrics} (rotations and reflections) of a rectangle.
Reflections: $T_x, T_y$, Rotations: $I, R_\pi$
  \item $G_3 = $ Rotation subgroup of $D_8$.
\end{itemize}

Check $G_1 \equiv G_3$

Note: $G_1, G_3$ are cyclic groups of order 4. Let $a \in D_8$ be a rotation of order 4. Then $G_3 = \{e,a,a^2, a^3\}$

Define a map $f:G_1 \rightarrow G_3$ by 

\begin{equation*}
  \begin{matrix}
    f(1)=e & f(-1) = a^2 \\
    f(i)=a & f(-i) = a^3
  \end{matrix}
  \label{}
\end{equation*}

Note: $f(i^n) = a^n$ for $\forall n \in \mathbb{Z}$

We can see $f$ is a bijection.

And: 
\begin{align*}
  f(i^ni^c) &= f(i^{n+c})\\
  &= a^{n+c} \\
  &= a^n a^c \\
  &= f(i^n)f(i^c)
\end{align*}

Hence, $f$ \emph{respects} multiplication so $G_1 \equiv G_3.$

\begin{proposition}
  Let $G$ and $H$ be groups 
  \begin{enumerate}
    \item If $|G| \neq |H|$ then $G$ and $H$ are not isomorphic
    \item If $G$ is abelian and $H$ is not abelian then $G \not\equiv H$
    \item If there exists $k \in \mathbb{N}$ such that $G$ and $H$ have disjoint numbers of elements of order $k$, then $G \not\equiv H$
  \end{enumerate}
\end{proposition}

\underline{Warning}

There do exist pairs of groups $G, H$ which passes the three above checks but which are not isomorphic.

\begin{proof}
  \begin{itemize}
    \item Any isomorphism is a bijection
    \item Hwk 1: 
      if $f : G \rightarrow H$ is an isomorphism then $f(g_1)$ commutes with $f(g_2) \iff g_1$ commutes with $g_2$ $\forall g_1,g_2 \in G$
    \item Hwk 1: 
      If $f:G\rightarrow H$ is an isomorphism then $ord(f(g))=ord(g)$ $\forall g\in G$
  \end{itemize}
\end{proof}

\underline{Examples:}
\begin{enumerate}
  \item $G=S_4, H=D_8$ - Disjoint orders so not isomorphic.
  \item $G=S_3, H=C_6=\{1,w,\ldots,w^5\}$ where $w=e^{\frac{2 \pi i}{6}}$ - $H$ is abelian, but $G$ is not, so not isomorphic.
  \item $G=C_4=\{1,i,-1,-i\}$, $H=\{I,R_\pi,T_x,T_y\}$ (the symmetry group of a rectangle). Orders of $G$ are 1,4,2,4, but the orders of $H$ are 1,2,2,2. We have disjoint numbers of order 2,4, so not isomorphic.
  \item $G=(\mathbb{R}, +)$, $H=(\mathbb{R}\backslash\{0\}, \times).$ -1 is an element of order 2 in $H$ but $G$ has no elements of order 2 - not isomorphic.
\end{enumerate}

\begin{proposition}
  \begin{enumerate}
    \item Let $G$ be a cyclic group of order $n$. Then $G \equiv \mathbb{Z}_n$
    \item Let $G$ be an infinite cyclic group. Then $G \equiv \mathbb{Z}$.
  \end{enumerate}
\end{proposition}

\begin{proof}
  \begin{enumerate}
    \item Let $G=\langle g \rangle $, where $G$ has order $n$. Define a map $f : \mathbb{Z}_n \rightarrow G$ by 
      $$f(i)=g^i \quad \text{for } i=0,\ldots,n-1$$
    Clearly $f$ is a bijection.
    \begin{enumerate}
      \item Suppose that $i+j<n$ for $i,j \in \{0,\ldots,n-1\}.$ Then $f(i+j)=g^{i+j}=g^i g^j = f(i)f(j)$.
      \item Suppose that $i+j \geq n$ The, $i+j = i+j-n$ in $\mathbb{Z}_n$ and $i+j-n \in \{0,\ldots,n-1\}$ so $f(i+j)=f(i+j-n)=g^{i+j-n}=g^{i+j}$ (since $g^n=e$).
        And this is $g^i g^j = f(i) f(j)$.
    \end{enumerate}
  \item Let $G=\langle g \rangle$ where $g$ has infinite order. Define $f : \mathbb{Z} \rightarrow G$ by $f(i)=g^i$ $\forall i \in \mathbb{Z}$. Clearly $f$ is a bijection and $f(i+j)=g^{i+j}=g^i g^j = f(i) f(j)$
  \end{enumerate}
\end{proof}

\begin{remark}
  It follows that $\mathbb{Z}_n \equiv C_n$ (complex $n^{th}$ roots of unity) for all $n$.
\end{remark}

\section{Even and odd permutations}

Consider the polynomial $\Delta_3 = (x_1 - x_2)(x_1 - x_3) (x_2 - x_3)$

Let $g \in S_3$.

Consider the effort of applying $g$ to the variables $x_1, x_2, x_3$ in $\Delta_3$.

$g_(\Delta_3) = (x_{g(1)} - x_{g(2)})(x_{g(1)} - x_{g(3)})(x_{g(2)} - x_{g(3)})$. 

We see that $g(\Delta_3)=I\Delta_3$.

More generally, let $\Delta_n = \prod_{i<j\leq n} (x_i - x_j ) $ Then if $g \in S_n$, we can apply $g$ to $\Delta_n$ by 
$$g(\Delta_n) = \prod_{i<j\leq n}(x_{g(i)} - x_{g(j)})$$

We have $g(\Delta_n) = I \Delta_n$

\begin{definition}
  The \emph{signature} of $g$, $\sgn(g)$, is 
  \begin{align*}
    +1 \quad \text{ if } g(\dtn) &= \dtn \\
    -1 \quad \text{ if } g(\dtn) &= -\dtn
  \end{align*}
\end{definition}

The \emph{parity} of $g$: If $\sgn(g)=+1$, say that $g$ is \emph{even}.
If $\sgn(g)=-1$, say that $g$ is \emph{odd}.

\underline{Example}
In $S_3$, we have that 

$id,\,(1 2 3), (1 3 2)$ are even.

$(1 2),\, (1 3),\,(2 3)$ are odd.

\underline{Observation}

The even permutation is $S_3$ form a subgroup (?)

\begin{proposition}
  \label{prp:sgn}
  \begin{enumerate}
    \item $\sgn(gh)=\sgn(g)\sgn(h)$ for all $g,h \in S_n$.
    \item $\sgn(g^{-1})=\sgn(g)$ for all $g \in S_n$.
    \item if $\tau=(i j)$ is a 2-cycle, then $\sgn(\tau)=-1$.
  \end{enumerate}
\end{proposition}

\begin{proof}
  \begin{enumerate}
    \item Let $g,h \in S_n$,
      \begin{align*}
        gh(\dtn) &= g(h(\dtn)) \\
                 &= g(\sgn(h)\dtn) \\
                 &= \sgn(h)(g(\dtn)) \\
                 &= \sgn(h) \sgn(g) \dtn
      \end{align*}
      So 
      \begin{align*}
        \sgn(gh) &= \sgn(h)\sgn(g) \\
                 &= \sgn(g)\sgn(h)
      \end{align*}

    \item We have $g g^{-1}=e$, and clearly $\sgn(e) = +1$. 

      So $\sgn(g g^{-1})=\sgn(g)\sgn(g^{-1})=+1$ (by (1)).

      So $\sgn(g^{-1})=\sgn(g)^{-1}$ by (1).

    \item Apply $(ij)(\dtn):$
      We count the number $N$ of brackets in $\dtn$ which are sent by $(ij)$ to things of the form $(x_k - x_i)$ where $k>i$ 

      Then $\sgn\left( (ij) \right)=(-1)^N$

      Count $p,q$ $p<q$, but $(ij)(p)>(ij)(q)$

      Cases: 
      \begin{enumerate}
        \item $i=p, j=q$ - (1 possibilty)
        \item $p=i,q<j$ - possibilties $q \in \left\{ i+1, \ldots, j-1 \right\}$
        \item $i<p,q=i$ - possibilties $p \in \left\{ i+1, \ldots , j-1 \right\}$
      \end{enumerate}
      So there are $2(j-i-2) +1$ possibilties in total.

      So $N=2(j-i-2)+1$ is odd. 

      Hence, $\sgn\left( (ij) \right)=-1$
  \end{enumerate}
\end{proof}

\begin{proposition}
  \label{prp:rcycle}
  Let $c=(a_1 \ldots a_r)$ by an $r$-cycle. 

  Then $c$ is a product of $r-1$ 2-cycles.
\end{proposition}

\begin{proof}
  Easy to check that
    $c=(a,a_n)\ldots(a,a_5)(a,a_4)(a,a_3)(a,a_2)$
\end{proof}

\begin{corollary}
  An $r$-cycle is even when $r$ is odd and even when $r$ is even.
\end{corollary}

\begin{proposition}
  \begin{enumerate}
    \item Every permutation can be written as a product of 2-cycles.
    \item If $g$ is a product of disjoint cycles of lengths $r_1, \ldots, l_j$

      Then the signature of $g$ is $\prod_{i} (-1)^{r_i - 1}$
  \end{enumerate}
\end{proposition}

\begin{proof}
  We saw last year that any permutation is a products of disjoint cycles.

  Each of the cycles is a product of 2-cycles by Prop. \ref{prp:rcycle} and so $g$ is too.

  (2) Follows from Prop. \ref{prp:rcycle} and Prop. \ref{prp:sgn}.1.
\end{proof}

Rule of thumb:

A permutation is even if it has an even number of cycles of even length

It is odd if it has an odd number of cycles of even length.

\underline{Example}

$$(1 2 3 4)(6 7 9)(8 4 10) \in S_{10}$$ is odd (one cycle of even length)

\section{Alternating Groups}

Define $A_n=\left\{ g \in S_n | \sgn(g) = +1 \right\}$ (The even permutations
 in $S_n$.

\begin{theorem}
  $A_n$ is a subgroup of $S_n$. If $n>1$, then $|A_n|=\frac{1}{2}n!$
  \label{thm:subgroup}
\end{theorem}

\begin{proof}
  $A_n$ is non-empty since it contains $e$. $A_n$ is closed under the group operation by Prop. \ref{prp:sgn}.1 and under inverses by Prop. \ref{prp:sgn}.2.

  Consider the coset $(12) A_n$. We know this coset has the same size as $A_n.$ Since $(1 2)$ is odd, we have
\begin{equation*}
  A_n \cap (1 2) A_n  = \emptyset
\end{equation*}

I claim that $A_n \cup (1 2) A_n $ is $S_n$. 

Let $g \in S_n$.

If $\sgn(g) = +1$ then $g \in A_n$.
If $\sgn(g) = -1$ then $\sgn\left( (12) g \right)= +1$.

So $(12)g \in A_n$

But $(1 2)\left( (1 2) g \right)=g$, and so $g \in (1 2)A_n$.

Hence, $|A_n|=\frac{1}{2}|S_n|=\frac{1}{2}n!$

\end{proof}

\textbf{Examples:}\\
$A_2 = \{ id \}$, $A_3 = \{id, (123), (132)\}$.

What about $A_4$? We know that $|A_4|=\frac{1}{2}4=12$.

The cycle structures in $S_4$ are:
$(1,1,1,1)$ - 1 element, $(2,2)$ - 3 elements, $(3,1)$ - 8 elements.
$A_5$ has order $\frac{1}{5}5! = 60$.

The even cycle shapes are:
$(1,1,1,1,1)$ - 1 element, $(2,2,1)$ - 15 elements, $(3,1,1)$ - 20 elements, $(5)$ - 24 elements. These four numbers add up to 60.

\emph{Exercise:} Do this for $A_6$.

\section{Direct products}
Recall that if we have set $X_1, \ldots, X_n$, the Cartesian product 
$X_1 \times X_2 \times \ldots \times X_n$ is the set of tuples $\{(x_1, \ldots,x_n) : x_i \in X_i\}$

What if the sets are actually groups?

Let $G_1 , \ldots, G_n$ be groups. Then we can define a binary operation on their Cartesian Product by
$$(g_1, \ldots, g_n) * (h_1, \ldots , h_n) = (g_1h_1, g_2h_2, \ldots, g_nh_n).$$
\begin{proposition}
Under this binary operation, $G_1 \times \ldots \times G_n$ is a group. This is called the \textit{direct product} of $G_1 , \ldots, G_n$.
\end{proposition}

\begin{proof}
Check the group axioms:
\begin{itemize}
  \item 
    \textbf{Associativity:}
  \begin{align*}
    & ((g_1, \, \ldots \, , g_n)(h_1,\, \ldots, \, h_n)) (k_1, \, \ldots , \, k_n) & \\
    &= (g_1 h_1, g_2 h_2, \ldots, g_n h_n) (k_1, \ldots , k_n) & \\
    &= ((g_1 h_1)k_1, \ldots , (h_n h_n)k_n), & 
      \text{each } G_i \text{ associative}\\
    &= (g_1, \ldots, g_n) (h_1 k_1, \ldots, h_n k_n) & \\
    &= (g_1, \ldots, g_n) ((h_1, \ldots, h_n) (h_1, \ldots, h_n)) &
  \end{align*}
\item
  \textbf{Identity:}

Let $e_i$ be the identity of $G_i$ for all $i$. Then $(e_1, \ldots, e_n)$ is an identity for $G_1 \times \ldots \times G_n$
\item
  \textbf{Inverses:}

The element $(g_1, \ldots , g_n)$ has the inverse $(g_1^{-1}, \ldots g_n^{-1})$.

\end{itemize}

\end{proof}
\emph{Example}: Consider the group $C_1 \times C_2$ which has elements $(1,1), (1,-1),(-1,1),(-1,-1)$

\begin{table}[]

\centering
\label{my-label}
\begin{tabular}{l|llll}
  & e & a & b & c \\ 
  \hline
e & e & a & b & c \\
a & a & e & c & b \\
b & b & c & e & a \\
c & c & b & a & e
\end{tabular}
\end{table}

Recall the group of symmetries of the rectangle, $\{I, R_\pi, T_x, T_y\}$.

This has group table:

\begin{table}[h]
\centering
\label{my-label}
\begin{tabular}{l|llll}
       & I      & $R_\pi$ & $T_x$   & $T_y$   \\
       \hline
I      & I      & $R_\pi$ & $T_x$   & $T_y$   \\
$R_\pi$ & $R_\pi$ & I      & $T_y$   & $T_X$    \\
$T_x$   & $T_x$   & $T_y$   & I      & $R_\pi$ \\
$T_\pi$ & $T_y$   & $T_x$   & $R_\pi$ & I     
\end{tabular}
\end{table}

So we have an isomorphism given by

$e \rightarrow I$\\
$a \rightarrow R\pi$\\
$b \rightarrow T_x$\\
$c \rightarrow T_y$\\

Recall from M1P2 that $S_4$ has a subgroup $$\{id,  (12)(34)=x, (13)(24)=y, (14)(23)=z\}.$$

This has group table 
\begin{table}[]
\centering
\begin{tabular}{l|llll}
   & id & x  & y  & z \\
   \hline
id & id & x  & y  & z \\
x  & x  & id & z  & y \\
y  & y  & z  & id & x \\
z  & z  & y  & x  & id 
\end{tabular}
\end{table}

So this too is isomorphic to $C_2 \times C_2$, via $e \rightarrow id, a \rightarrow id, a \rightarrow x, b \rightarrow y, c \rightarrow z$.

\emph{Example}: $C_2 \times C_2 \times C_2$, which has elements $(\pm 1, \pm 1, \pm 1)$. We see there are 8 elements and $x^2=(1,1,1)=e$ for any $x$.

\begin{proposition}
\begin{itemize}
\item $$|G_1 \times \ldots \times G_n| =|G_1||G_2|\ldots|G_n|.$$
\item If \emph{all} of the gorups $G_i$ are abelian, then so is $G_1 \times \ldots \times G_n$. But if \emph{any} of $G_1, \ldots G_n$ is not abelian then netiher is $G_1 \times \ldots \times G_n$.
\item Let $(g_1, \ldots g_n) \in G_1 \times \ldots \times G_n$. Then $ord (g_1, \ldots , g_n) = lcm (ord g, \ldots , ord g_n)$.
Proof not known.
\end{itemize}
\end{proposition}



\end{document}
