\documentclass{article}
\usepackage{amsmath}
\usepackage[utf8]{inputenc}
\usepackage{amssymb}
\usepackage{array}
\usepackage{amsthm}
\usepackage{parskip}

\newtheorem{theorem}{Theorem}[section]

\theoremstyle{definition}
\newtheorem{definition}{Definition}[section]

\newtheorem{lemma}[theorem]{Lemma}
\newtheorem{proposition}[theorem]{Proposition}
\newtheorem{Corollary}[theorem]{Corollary}
\newtheorem{remark}{Remark}

\title{Algebra II}
\author{John R. Britnell - 655\\
        Problems class: 12:00pm\\
        Office hour: Mon 2:00pm}
\date{October 2015}

\begin{document}

\maketitle

\section{Groups}
A group is a set $G$ with a binary operation $*$ such that 
\begin{itemize}
\item Associativity: $(x * y)*z = x * (y * z)$ for all $x,y,z \in G$.
\item Identity: There is $e \in G$ such that $e * x = x * e$ for all $x \in G$.
\item Inverses: For all $x \in G$ there exists $y \in G$ such that $x * y = yz * y = e$
\end{itemize}

\subsection{Some groups}
\begin{enumerate}
\item
\begin{itemize}
\item $\mathbb{Z}_n$, integers modulo $n$ under $+$.
\item $\mathbb{Z}$, integers under $+$.
\end{itemize}
These are \emph{cyclic groups}.
\item
\begin{itemize}
\item $D_{2n}$, $n \geq 3$ dihedral group (of rotations and reflections) of a regular $n$-gon.
\item We also have $D_\infty$ the \emph{infinite dihedral group}.

Take the ``polygon''.

What are the ``rotations''? These are shifts: move each vertex some number $k$ of places to the right. (if $k$ is -ve we shift left instead). 

``Reflections'' really are reflections - through a vertical axis, either through a vertex or the midpoint of an ``edge''.

The ``rotations'' and ``reflections'' form a group under composition. This is $D_\infty$. The subgroup of rotations is an infinite cyclic group, generated by $R_1$

\end{itemize}
\item Symmetric groups $S_n$. Permutations of $\{1, \cdots, n\}$ under composition. $|S_n| = n!$
More generally if $\Omega$ is a set then $Sym(\Omega)$, the symmetric group on $\Omega$ is the group of all permutations of $\Omega$.

\item Let $F$ be a field. Then $GL_n(F)$, the \emph{General Linear Group of degree $n$ over $F$}, is the set of \emph{invertible} $n \times n$ matrices with entries from $F$. It is a group under matrix multiplication. 
\end{enumerate}

\section{Motivating Example}

\begin{table}[h]
\centering
\begin{tabular}{l|llllll}
$s_3$  & e     & (123) & (132) & (12)  & (13)  & (23)  \\
\hline
e     & e     & (123) & (132) & (12)  & (13)  & (23)   \\
(123) & (123) & (132) & e     & (13)  & (23)  & (12)   \\
(132) & (132) & e     & (123) & (23)  & (12)  & (13)   \\
(12)  & (12)  & (23)  & (13)  & e     & (132) & (123)  \\
(13)  & (13)  & (12)  & (23)  & (123) & e     & (132)  \\
(23)  & (23)  & (13)  & (12)  & (132) & (123) & e      \\
\end{tabular}
\end{table}

These tables coincide if we make the equivalence:
$$ 
\begin{matrix}
  e & \leftrightarrow & e \\
  R & \leftrightarrow & (123) \\
  R^{-1} & \leftrightarrow & (132) \\
  U & \leftrightarrow & (12) \\
  V & \leftrightarrow & (13) \\
  W & \leftrightarrow & (23) \\
\end{matrix}
$$


%\begin{equation*}
%e &= e \\
%R &= (123) \\
%R^{-1} &= (132) \\
%U &= (12) \\
%V &= (13) \\
%W &= (23)
%\end{equation*}

From an algebraic point of view these two groups are ``the same''. We say they are \textit{isomorphic}.

Here's a formal definition.\\


\begin{definition}
Let $G$ and $H$ be groups, and let $f : G \rightarrow H$ be a function. We say that $f$ is an \emph{isomorphism} if:
\begin{enumerate}
\item $f$ is a bijection.
\item $f(g_1 * g_2)=f(g_1) * f(g_2) $ for all $g_1, g_2 \in G$.

Note that in $f(g_1 * g_2)$ the multiplication is happening in $G$, but in $f(g_1) * f(g_2)$ the multiplication is in $H$. If condition (2) is satisfied, we say that f \emph{respects} $*$.

$G$ and $H$ are \emph{isomorphic} if an isomorphism $G \rightarrow H$ exists.
\end{enumerate}
\end{definition}

\begin{itemize}
  \item 
$G \equiv H$ if $G$ and $H$ are isomorphic and $f$ is the isomorphism between them
  \item
$G \equiv H$ if $G$ and $H$ are isomorhpic via some isomorphism (particular isomorphism not mentioned)
  \item 
$G \not\equiv H$ if $G$ and $H$ are not isomorphic.
\end{itemize}

\emph{Examples}: Determine which pairs are isomorphic:

\begin{itemize}
  \item $S_2=\{e, (1, 2)\}$
  \item $\mathbb{Z}_2=\{0,1\}$ operations: + mod 2
  \item $C_2=\{1,-1\}$ operations: multiplication
\end{itemize}
$C_n$ represents $n^{th}$ roots of unity.

Look at Group Tables:

\begin{table}[h]
  \centering
\label{my-label}
\begin{tabular}{l|ll}
$s_2$  & e     & (1,2) \\
\hline
e     & e     & (1,2) \\
(1,2) & (1,2) & e    
\end{tabular}
\end{table}

\begin{table}[h]
  \centering
\label{my-label}
\begin{tabular}{l|ll}
  $\mathbb{Z}_2$  & 0  & 1\\
\hline
0     & 0     & 1 \\
1 & 1 & 0    
\end{tabular}
\end{table}

\begin{table}[h]
  \centering
\label{my-label}
\begin{tabular}{l|ll}
  $C_2$  & 1  & -1\\
\hline
1     & 1  & -1 \\
-1    & -1 & 1    
\end{tabular}
\end{table}
All groups are isomorphic, which can be shown by `relabelling': 

\begin{table}[h]
  \centering
\label{my-label}
\begin{tabular}{lllll}
  $S_2$  &  & $\mathbb{Z}_2$ & & $C_2$\\
\hline
e     & $\leftrightarrow$ & 0 & $\leftrightarrow$ & 1 \\
(1,2) & $\leftrightarrow$ & 1 & $\leftrightarrow$ & -1    
\end{tabular}
\end{table}

Are $\mathbb{Z}_3=\{0,1,2\}$ and $C_3=\{1,w,w^2\}$, where $w=e^{\frac{2\pi i}{3}}$ isomorphic?

\underline{Yes}, an isomorphism is:
\begin{equation}
  f : 
       \begin{matrix}
        0 & \mapsto & 1 \\
        1 & \mapsto & w \\
        2 & \mapsto & w^2 \\
        \mathbb{Z}_3 & \mapsto & C_3
      \end{matrix} 
\end{equation}

Another:
\begin{equation}
  \hat{f} :
       \begin{matrix}
        0 & \mapsto & 1 \\
        1 & \mapsto & w^2 \\
        2 & \mapsto & w \\
      \end{matrix} 
\end{equation}

\begin{remark}
\begin{enumerate}
  \item Let $G$ be finite and let $G \equiv H$ then $|G|=|H|$ i.e. sets are the same size, clearly since isomorphism is a bijection.
  \item Let $G$ have identiy $e_G$ and $H$ have identity $e_H$. Suppose $G \equiv H$. Then, $f(e_G)=e_H$.
 
\end{enumerate}
\end{remark}


\newpage
\section{Alternating Groups}
\textbf{Examples:}\\
$A_2 = \{ id \}$, $A_3 = \{id, (123), (132)\}$.

What about $A_4$? We know that $|A_4|=\frac{1}{2}4=12$.

The cycle structures in $S_4$ are:
$(1,1,1,1)$ - 1 element, $(2,2)$ - 3 elements, $(3,1)$ - 8 elements.
$A_5$ has order $\frac{1}{5}5! = 60$.

The even cycle shapes are:
$(1,1,1,1,1)$ - 1 element, $(2,2,1)$ - 15 elements, $(3,1,1)$ - 20 elements, $(5)$ - 24 elements. These four numbers add up to 60.

\emph{Exercise:} Do this for $A_6$.

\section{Direct products}
Recall that if we have set $X_1, \ldots, X_n$, the Cartesian product 
$X_1 \times X_2 \times \ldots \times X_n$ is the set of tuples $\{(x_1, \ldots,x_n) : x_i \in X_i\}$

What if the sets are actually groups?

Let $G_1 , \ldots, G_n$ be groups. Then we can define a binary operation on their Cartesian Product by
$$(g_1, \ldots, g_n) * (h_1, \ldots , h_n) = (g_1h_1, g_2h_2, \ldots, g_nh_n).$$
\begin{proposition}
Under this binary operation, $G_1 \times \ldots \times G_n$ is a group. This is called the \textit{direct product} of $G_1 , \ldots, G_n$.
\end{proposition}

\begin{proof}
Check the group axioms:
\begin{itemize}
  \item 
    \textbf{Associativity:}
  \begin{align*}
    & ((g_1, \, \ldots \, , g_n)(h_1,\, \ldots, \, h_n)) (k_1, \, \ldots , \, k_n) & \\
    &= (g_1 h_1, g_2 h_2, \ldots, g_n h_n) (k_1, \ldots , k_n) & \\
    &= ((g_1 h_1)k_1, \ldots , (h_n h_n)k_n), & 
      \text{each } G_i \text{ associative}\\
    &= (g_1, \ldots, g_n) (h_1 k_1, \ldots, h_n k_n) & \\
    &= (g_1, \ldots, g_n) ((h_1, \ldots, h_n) (h_1, \ldots, h_n)) &
  \end{align*}
\item
  \textbf{Identity:}

Let $e_i$ be the identity of $G_i$ for all $i$. Then $(e_1, \ldots, e_n)$ is an identity for $G_1 \times \ldots \times G_n$
\item
  \textbf{Inverses:}

The element $(g_1, \ldots , g_n)$ has the inverse $(g_1^{-1}, \ldots g_n^{-1})$.

\end{itemize}

\end{proof}
\emph{Example}: Consider the group $C_1 \times C_2$ which has elements $(1,1), (1,-1),(-1,1),(-1,-1)$

\begin{table}[]

\centering
\label{my-label}
\begin{tabular}{l|llll}
  & e & a & b & c \\ 
  \hline
e & e & a & b & c \\
a & a & e & c & b \\
b & b & c & e & a \\
c & c & b & a & e
\end{tabular}
\end{table}

Recall the group of symmetries of the rectangle, $\{I, R_\pi, T_x, T_y\}$.

This had group t

\begin{table}[]
\centering
\label{my-label}
\begin{tabular}{l|llll}
       & I      & $R_\pi$ & $T_x$   & $T_y$   \\
       \hline
I      & I      & $R_\pi$ & $T_x$   & $T_y$   \\
$R_\pi$ & $R_\pi$ & I      & $T_y$   & $T_X$    \\
$T_x$   & $T_x$   & $T_y$   & I      & $R_\pi$ \\
$T_\pi$ & $T_y$   & $T_x$   & $R_\pi$ & I     
\end{tabular}
\end{table}

So we have an isomorphism given by

$e \rightarrow I$\\
$a \rightarrow R\pi$\\
$b \rightarrow T_x$\\
$c \rightarrow T_y$\\

Recall from M1P2 that $S_4$ has a subgroup $$\{id,  (12)(34)=x, (13)(24)=y, (14)(23)=z\}.$$

This has group table 
\begin{table}[]
\centering
\begin{tabular}{l|llll}
   & id & x  & y  & z \\
   \hline
id & id & x  & y  & z \\
x  & x  & id & z  & y \\
y  & y  & z  & id & x \\
z  & z  & y  & x  &  
\end{tabular}
\end{table}

So this too is isomorphic to $C_2 \times C_2$, via $e \rightarrow id, a \rightarrow id, a \rightarrow x, b \rightarrow y, c \rightarrow z$.

\emph{Example}: $C_2 \times C_2 \times C_2$, which has elements $(\pm 1, \pm 1, \pm 1)$. We see there are 8 elements and $x^2=(1,1,1)=e$ for any $x$.

\begin{proposition}
\begin{itemize}
\item $$|G_1 \times \ldots \times G_n| =|G_1||G_2|\ldots|G_n|.$$
\item If \emph{all} of the gorups $G_i$ are abelian, then so is $G_1 \times \ldots \times G_n$. But if \emph{any} of $G_1, \ldots G_n$ is not abelian then netiher is $G_1 \times \ldots \times G_n$.
\item Let $(g_1, \ldots g_n) \in G_1 \times \ldots \times G_n$. Then $ord (g_1, \ldots , g_n) = lcm (ord g, \ldots , ord g_n)$.
Proof not known.
\end{itemize}
\end{proposition}



\end{document}
