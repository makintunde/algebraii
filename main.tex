\documentclass{article}
\usepackage{amsmath}
\usepackage[utf8]{inputenc}
\usepackage{amssymb}
\usepackage{array}
\usepackage{amsthm}

\newtheorem{theorem}{Theorem}[section]

\theoremstyle{definition}
\newtheorem{definition}{Definition}[section]

\newtheorem{lemma}[theorem]{Lemma}
\newtheorem{proposition}{Proposition}
\newtheorem{corollary}[theorem]{Corollary}
\newtheorem*{remark}{Remark}
\newtheorem*{exmp}{Example}


\newcommand{\dtn}{\Delta_n}
\newcommand{\gene}[1]{\langle #1 \rangle}
\newcommand{\nsg}[2]{#1 \trianglelefteq #2}
\newcommand{\func}[3]{#1 : #2 \rightarrow #3}
\newcommand{\ism}{\cong}
\DeclareMathOperator{\sgn}{sgn}
\DeclareMathOperator{\Ima}{Im}
\DeclareMathOperator{\Sym}{Sym}
\DeclareMathOperator{\ord}{\text{ord}}


\title{Algebra II}
\author{John R. Britnell - 655\\
        Problems class: Wednesday 2:00pm\\
        Office hour: Mon 2:00pm}
\date{October 2015}

\begin{document}

\maketitle

\section{Groups}
A group is a set $G$ with a binary operation $*$ such that 
\begin{itemize}
\item Associativity: $(x * y)*z = x * (y * z)$ for all $x,y,z \in G$.
\item Identity: There is $e \in G$ such that $e * x = x * e$ for all $x \in G$.
\item Inverses: For all $x \in G$ there exists $y \in G$ such that $x * y = yz * y = e$
\end{itemize}

\subsection{Some groups}
\begin{enumerate}
\item
\begin{itemize}
\item $\mathbb{Z}_n$, integers modulo $n$ under $+$.
\item $\mathbb{Z}$, integers under $+$.
\end{itemize}
These are \emph{cyclic groups}.
\item
\begin{itemize}
\item $D_{2n}$, $n \geq 3$ dihedral group (of rotations and reflections) of a regular $n$-gon.
\item We also have $D_\infty$ the \emph{infinite dihedral group}.

Take the ``polygon''.

What are the ``rotations''? These are shifts: move each vertex some number $k$ of places to the right. (if $k$ is -ve we shift left instead). 

``Reflections'' really are reflections - through a vertical axis, either through a vertex or the midpoint of an ``edge''.

The ``rotations'' and ``reflections'' form a group under composition. This is $D_\infty$. The subgroup of rotations is an infinite cyclic group, generated by $R_1$

\end{itemize}
\item Symmetric groups $S_n$. Permutations of $\{1, \ldots, n\}$ under composition. $|S_n| = n!$
More generally if $\Omega$ is a set then $\Sym(\Omega)$, the symmetric group on $\Omega$ is the group of all permutations of $\Omega$.

\item Let $F$ be a field. Then $GL_n(F)$, the \emph{General Linear Group of degree $n$ over $F$}, is the set of \emph{invertible} $n \times n$ matrices with entries from $F$. It is a group under matrix multiplication. 
\end{enumerate}

\section{Motivating Example}

\begin{table}[h]
\centering
\begin{tabular}{l|llllll}
$s_3$  & e     & (123) & (132) & (12)  & (13)  & (23)  \\
\hline
e     & e     & (123) & (132) & (12)  & (13)  & (23)   \\
(123) & (123) & (132) & e     & (13)  & (23)  & (12)   \\
(132) & (132) & e     & (123) & (23)  & (12)  & (13)   \\
(12)  & (12)  & (23)  & (13)  & e     & (132) & (123)  \\
(13)  & (13)  & (12)  & (23)  & (123) & e     & (132)  \\
(23)  & (23)  & (13)  & (12)  & (132) & (123) & e      \\
\end{tabular}
\end{table}

These tables coincide if we make the equivalence:
$$ 
\begin{matrix}
  e & \leftrightarrow & e \\
  R & \leftrightarrow & (123) \\
  R^{-1} & \leftrightarrow & (132) \\
  U & \leftrightarrow & (12) \\
  V & \leftrightarrow & (13) \\
  W & \leftrightarrow & (23) \\
\end{matrix}
$$

From an algebraic point of view these two groups are ``the same''. We say they are \textit{isomorphic}.

Here's a formal definition.\\


\begin{definition}
Let $G$ and $H$ be groups, and let $f : G \rightarrow H$ be a function. We say that $f$ is an \emph{isomorphism} if:
\begin{enumerate}
\item $f$ is a bijection.
\item $f(g_1 * g_2)=f(g_1) * f(g_2) $ for all $g_1, g_2 \in G$.

Note that in $f(g_1 * g_2)$ the multiplication is happening in $G$, but in $f(g_1) * f(g_2)$ the multiplication is in $H$. If condition (2) is satisfied, we say that f \emph{respects} $*$.

$G$ and $H$ are \emph{isomorphic} if an isomorphism $G \rightarrow H$ exists.
\end{enumerate}
\end{definition}

\begin{itemize}
  \item 
$G \ism H$ if $G$ and $H$ are isomorphic and $f$ is the isomorphism between them.
  \item
$G \ism H$ if $G$ and $H$ are isomorhpic via some isomorphism (particular isomorphism not mentioned)
  \item 
$G \not\ism H$ if $G$ and $H$ are not isomorphic.
\end{itemize}

\emph{Examples}: Determine which pairs are isomorphic:

\begin{itemize}
  \item $S_2=\{e, (1, 2)\}$
  \item $\mathbb{Z}_2=\{0,1\}$ operations: $+\mod 2$
  \item $C_2=\{1,-1\}$ operations: multiplication
\end{itemize}
$C_n$ represents $n^{th}$ roots of unity.

Look at the group tables of the above:

\begin{table}[h]
  \centering
\label{my-label}
\begin{tabular}{l|ll}
$S_2$  & e     & (1,2) \\
\hline
e     & e     & (1,2) \\
(1,2) & (1,2) & e    
\end{tabular}
\end{table}

\begin{table}[h]
  \centering
\label{my-label}
\begin{tabular}{l|ll}
  $\mathbb{Z}_2$  & 0  & 1\\
\hline
0     & 0     & 1 \\
1 & 1 & 0    
\end{tabular}
\end{table}

\begin{table}[h]
  \centering
\label{my-label}
\begin{tabular}{l|ll}
  $C_2$  & 1  & -1\\
\hline
1     & 1  & -1 \\
-1    & -1 & 1    
\end{tabular}
\end{table}
All groups are isomorphic, which can be shown by `relabelling': 

\begin{table}[h]
  \centering
\label{my-label}
\begin{tabular}{lllll}
  $S_2$  &  & $\mathbb{Z}_2$ & & $C_2$\\
\hline
e     & $\leftrightarrow$ & 0 & $\leftrightarrow$ & 1 \\
(1,2) & $\leftrightarrow$ & 1 & $\leftrightarrow$ & -1    
\end{tabular}
\end{table}

Are $\mathbb{Z}_3=\{0,1,2\}$ and $C_3=\{1,w,w^2\}$, where $w=e^{\frac{2\pi i}{3}}$ isomorphic?

\textbf{Yes}, an isomorphism is:
\begin{equation}
  f : \left\{ 
       \begin{matrix}
        0 & \mapsto & 1 \\
        1 & \mapsto & w \\
        2 & \mapsto & w^2 \\
        \mathbb{Z}_3 & \mapsto & C_3
      \end{matrix}
      \right.
\end{equation}

Another:
\begin{equation}
  \hat{f} : \left\{
       \begin{matrix}
        0 & \mapsto & 1 \\
        1 & \mapsto & w^2 \\
        2 & \mapsto & w \\
      \end{matrix} 
      \right.
\end{equation}

\begin{remark}\hfill
\begin{enumerate}
  \item Let $G$ be finite and let $G \ism H$ then $|G|=|H|$ i.e. sets are the same size, clearly since isomorphism is a bijection.
  \item Let $G$ have identiy $e_G$ and $H$ have identity $e_H$. Suppose $G \ism H$. Then, $f(e_G)=e_H$.
  \item `$\ism$' is an is an equivalence relation on groups.
   We have symmetry, hence order does not matter for isomorphism:
    \begin{itemize}
      \item[\textbf{Reflexivity}:] $G \ism G $
      \item[\textbf{Symmetry}:] $G \ism H \iff H \ism G$
      \item[\textbf{Transitivity}:] $G \ism H, H \ism H \implies G \ism K $ for all groups $G,H,K$
    \end{itemize}
\end{enumerate}
\end{remark}

\begin{exmp}
Which of these pairs of groups are isomorphic?
\begin{itemize}
  \item 
$G_1 = C_4$ = $\{1,-1,i,-i\}$
  \item
$G_2 = $ group of \emph{symmetrics} (rotations and reflections) of a rectangle.\\
Reflections: $T_x, T_y$, Rotations: $I, R_\pi$
  \item $G_3 = $ Rotation subgroup of $D_8$.
\end{itemize}
  
Check $G_1 \ism G_3$.
\end{exmp}

\textbf{Note}: $G_1, G_3$ are cyclic groups of order 4. Let $a \in D_8$ be a rotation of order 4. 
Then $G_3 = \{e,a,a^2, a^3\}$.\hfill\\

Define a map $f:G_1 \rightarrow G_3$ by 

\begin{equation*}
  \begin{matrix}
    f(1)=e & f(-1) = a^2 \\
    f(i)=a & f(-i) = a^3
  \end{matrix}
  \label{}
\end{equation*}

Note: $f(i^n) = a^n$ for $\forall n \in \mathbb{Z}$

We can see $f$ is a bijection.

And: 
\begin{align*}
  f(i^ni^c) &= f(i^{n+c})\\
  &= a^{n+c} \\
  &= a^n a^c \\
  &= f(i^n)f(i^c)
\end{align*}

Hence, $f$ \emph{respects} multiplication so $G_1 \equiv G_3.$

\begin{proposition}
  Let $G$ and $H$ be groups 
  \begin{enumerate}
    \item If $|G| \neq |H|$ then $G$ and $H$ are not isomorphic
    \item If $G$ is abelian and $H$ is not abelian then $G \not\equiv H$
    \item If there exists $k \in \mathbb{N}$ such that $G$ and $H$ have disjoint numbers of elements of order $k$, then $G \not\equiv H$
  \end{enumerate}
\end{proposition}

\textbf{Warning:} There do exist pairs of groups $G, H$ which passes the three above checks but which are not isomorphic.

\begin{proof}
  \begin{itemize}
    \item Any isomorphism is a bijection
    \item Hwk 1: 
      if $f : G \rightarrow H$ is an isomorphism then $f(g_1)$ commutes with $f(g_2) \iff g_1$ commutes with $g_2$ $\forall g_1,g_2 \in G$
    \item Hwk 1: 
      If $f:G\rightarrow H$ is an isomorphism then $ord(f(g))=ord(g)$ $\forall g\in G$
  \end{itemize}
\end{proof}

\underline{Examples:}
\begin{enumerate}
  \item $G=S_4, H=D_8$ - Disjoint orders so not isomorphic.
  \item $G=S_3, H=C_6=\{1,w,\ldots,w^5\}$ where $w=e^{\frac{2 \pi i}{6}}$ - $H$ is abelian, but $G$ is not, so not isomorphic.
  \item $G=C_4=\{1,i,-1,-i\}$, $H=\{I,R_\pi,T_x,T_y\}$ (the symmetry group of a rectangle). Orders of $G$ are 1,4,2,4, but the orders of $H$ are 1,2,2,2. We have disjoint numbers of order 2,4, so not isomorphic.
  \item $G=(\mathbb{R}, +)$, $H=(\mathbb{R}\backslash\{0\}, \times).$ -1 is an element of order 2 in $H$ but $G$ has no elements of order 2 - not isomorphic.
\end{enumerate}

\begin{proposition}
  \begin{enumerate}
    \item Let $G$ be a cyclic group of order $n$. Then $G \equiv \mathbb{Z}_n$
    \item Let $G$ be an infinite cyclic group. Then $G \equiv \mathbb{Z}$.
  \end{enumerate}
\end{proposition}

\begin{proof}
  \begin{enumerate}
    \item Let $G=\langle g \rangle $, where $G$ has order $n$. Define a map $f : \mathbb{Z}_n \rightarrow G$ by 
      $$f(i)=g^i \quad \text{for } i=0,\ldots,n-1$$
    Clearly $f$ is a bijection.
    \begin{enumerate}
      \item Suppose that $i+j<n$ for $i,j \in \{0,\ldots,n-1\}.$ Then $f(i+j)=g^{i+j}=g^i g^j = f(i)f(j)$.
      \item Suppose that $i+j \geq n$ The, $i+j = i+j-n$ in $\mathbb{Z}_n$ and $i+j-n \in \{0,\ldots,n-1\}$ so $f(i+j)=f(i+j-n)=g^{i+j-n}=g^{i+j}$ (since $g^n=e$).
        And this is $g^i g^j = f(i) f(j)$.
    \end{enumerate}
  \item Let $G=\langle g \rangle$ where $g$ has infinite order. Define $f : \mathbb{Z} \rightarrow G$ by $f(i)=g^i$ $\forall i \in \mathbb{Z}$. Clearly $f$ is a bijection and $f(i+j)=g^{i+j}=g^i g^j = f(i) f(j)$
  \end{enumerate}
\end{proof}

\begin{remark}
  It follows that $\mathbb{Z}_n \equiv C_n$ (complex $n^{th}$ roots of unity) for all $n$.
\end{remark}

\section{Even and odd permutations}

Consider the polynomial $\Delta_3 = (x_1 - x_2)(x_1 - x_3) (x_2 - x_3)$

Let $g \in S_3$.

Consider the effort of applying $g$ to the variables $x_1, x_2, x_3$ in $\Delta_3$.

$g_(\Delta_3) = (x_{g(1)} - x_{g(2)})(x_{g(1)} - x_{g(3)})(x_{g(2)} - x_{g(3)})$. 

We see that $g(\Delta_3)=I\Delta_3$.

More generally, let $\Delta_n = \prod_{i<j\leq n} (x_i - x_j ) $ Then if $g \in S_n$, we can apply $g$ to $\Delta_n$ by 
$$g(\Delta_n) = \prod_{i<j\leq n}(x_{g(i)} - x_{g(j)})$$

We have $g(\Delta_n) = I \Delta_n$

\begin{definition}
  The \emph{signature} of $g$, $\sgn(g)$, is 
  \begin{align*}
    +1 \quad \text{ if } g(\dtn) &= \dtn \\
    -1 \quad \text{ if } g(\dtn) &= -\dtn
  \end{align*}
\end{definition}

The \emph{parity} of $g$: If $\sgn(g)=+1$, say that $g$ is \emph{even}.
If $\sgn(g)=-1$, say that $g$ is \emph{odd}.

\underline{Example}
In $S_3$, we have that 

$id,\,(1 2 3), (1 3 2)$ are even.

$(1 2),\, (1 3),\,(2 3)$ are odd.

\underline{Observation}

The even permutation is $S_3$ form a subgroup (?)

\begin{proposition}
  \label{prp:sgn}
  \begin{enumerate}
    \item $\sgn(gh)=\sgn(g)\sgn(h)$ for all $g,h \in S_n$.
    \item $\sgn(g^{-1})=\sgn(g)$ for all $g \in S_n$.
    \item if $\tau=(i j)$ is a 2-cycle, then $\sgn(\tau)=-1$.
  \end{enumerate}
\end{proposition}

\begin{proof}
  \begin{enumerate}
    \item Let $g,h \in S_n$,
      \begin{align*}
        gh(\dtn) &= g(h(\dtn)) \\
                 &= g(\sgn(h)\dtn) \\
                 &= \sgn(h)(g(\dtn)) \\
                 &= \sgn(h) \sgn(g) \dtn
      \end{align*}
      So 
      \begin{align*}
        \sgn(gh) &= \sgn(h)\sgn(g) \\
                 &= \sgn(g)\sgn(h)
      \end{align*}

    \item We have $g g^{-1}=e$, and clearly $\sgn(e) = +1$. 

      So $\sgn(g g^{-1})=\sgn(g)\sgn(g^{-1})=+1$ (by (1)).

      So $\sgn(g^{-1})=\sgn(g)^{-1}$ by (1).

    \item Apply $(ij)(\dtn):$
      We count the number $N$ of brackets in $\dtn$ which are sent by $(ij)$ to things of the form $(x_k - x_i)$ where $k>i$ 

      Then $\sgn\left( (ij) \right)=(-1)^N$

      Count $p,q$ $p<q$, but $(ij)(p)>(ij)(q)$

      Cases: 
      \begin{enumerate}
        \item $i=p, j=q$ - (1 possibilty)
        \item $p=i,q<j$ - possibilties $q \in \left\{ i+1, \ldots, j-1 \right\}$
        \item $i<p,q=i$ - possibilties $p \in \left\{ i+1, \ldots , j-1 \right\}$
      \end{enumerate}
      So there are $2(j-i-2) +1$ possibilties in total.

      So $N=2(j-i-2)+1$ is odd. 

      Hence, $\sgn\left( (ij) \right)=-1$
  \end{enumerate}
\end{proof}

\begin{proposition}
  \label{prp:rcycle}
  Let $c=(a_1 \ldots a_r)$ by an $r$-cycle. 

  Then $c$ is a product of $r-1$ 2-cycles.
\end{proposition}

\begin{proof}
  Easy to check that
    $c=(a,a_n)\ldots(a,a_5)(a,a_4)(a,a_3)(a,a_2)$
\end{proof}

\begin{corollary}
  An $r$-cycle is even when $r$ is odd and even when $r$ is even.
\end{corollary}

\begin{proposition}
  \begin{enumerate}
    \item Every permutation can be written as a product of 2-cycles.
    \item If $g$ is a product of disjoint cycles of lengths $r_1, \ldots, l_j$

      Then the signature of $g$ is $\prod_{i} (-1)^{r_i - 1}$
  \end{enumerate}
\end{proposition}

\begin{proof}
  We saw last year that any permutation is a products of disjoint cycles.

  Each of the cycles is a product of 2-cycles by Prop. \ref{prp:rcycle} and so $g$ is too.

  (2) Follows from Prop. \ref{prp:rcycle} and Prop. \ref{prp:sgn}.1.
\end{proof}

Rule of thumb:

A permutation is even if it has an even number of cycles of even length

It is odd if it has an odd number of cycles of even length.

\underline{Example}

$$(1 2 3 4)(6 7 9)(8 4 10) \in S_{10}$$ is odd (one cycle of even length)

\section{Alternating Groups}

Define $A_n=\left\{ g \in S_n | \sgn(g) = +1 \right\}$ (The even permutations
 in $S_n$.

\begin{theorem}
  $A_n$ is a subgroup of $S_n$. If $n>1$, then $|A_n|=\frac{1}{2}n!$
  \label{thm:subgroup}
\end{theorem}

\begin{proof}
  $A_n$ is non-empty since it contains $e$. $A_n$ is closed under the group operation by Prop. \ref{prp:sgn}.1 and under inverses by Prop. \ref{prp:sgn}.2.

  Consider the coset $(12) A_n$. We know this coset has the same size as $A_n.$ Since $(1 2)$ is odd, we have
\begin{equation*}
  A_n \cap (1 2) A_n  = \emptyset
\end{equation*}

I claim that $A_n \cup (1 2) A_n $ is $S_n$. 

Let $g \in S_n$.

If $\sgn(g) = +1$ then $g \in A_n$.
If $\sgn(g) = -1$ then $\sgn\left( (12) g \right)= +1$.

So $(12)g \in A_n$

But $(1 2)\left( (1 2) g \right)=g$, and so $g \in (1 2)A_n$.

Hence, $|A_n|=\frac{1}{2}|S_n|=\frac{1}{2}n!$

\end{proof}

\textbf{Examples:}\\
$A_2 = \{ id \}$, $A_3 = \{id, (123), (132)\}$.

What about $A_4$? We know that $|A_4|=\frac{1}{2}4=12$.

The cycle structures in $S_4$ are:
$(1,1,1,1)$ - 1 element, $(2,2)$ - 3 elements, $(3,1)$ - 8 elements.
$A_5$ has order $\frac{1}{5}5! = 60$.

The even cycle shapes are:
$(1,1,1,1,1)$ - 1 element, $(2,2,1)$ - 15 elements, $(3,1,1)$ - 20 elements, $(5)$ - 24 elements. These four numbers add up to 60.

\emph{Exercise:} Do this for $A_6$.

\section{Direct products}
Recall that if we have set $X_1, \ldots, X_n$, the Cartesian product 
$X_1 \times X_2 \times \ldots \times X_n$ is the set of tuples $\{(x_1, \ldots,x_n) : x_i \in X_i\}$

What if the sets are actually groups?

Let $G_1 , \ldots, G_n$ be groups. Then we can define a binary operation on their Cartesian Product by
$$(g_1, \ldots, g_n) * (h_1, \ldots , h_n) = (g_1h_1, g_2h_2, \ldots, g_nh_n).$$
\begin{proposition}
Under this binary operation, $G_1 \times \ldots \times G_n$ is a group. This is called the \textit{direct product} of $G_1 , \ldots, G_n$.
\end{proposition}

\begin{proof}
Check the group axioms:
\begin{itemize}
  \item 
    \textbf{Associativity:}
  \begin{align*}
    & ((g_1, \, \ldots \, , g_n)(h_1,\, \ldots, \, h_n)) (k_1, \, \ldots , \, k_n) & \\
    &= (g_1 h_1, g_2 h_2, \ldots, g_n h_n) (k_1, \ldots , k_n) & \\
    &= ((g_1 h_1)k_1, \ldots , (h_n h_n)k_n), & 
      \text{each } G_i \text{ associative}\\
    &= (g_1, \ldots, g_n) (h_1 k_1, \ldots, h_n k_n) & \\
    &= (g_1, \ldots, g_n) ((h_1, \ldots, h_n) (h_1, \ldots, h_n)) &
  \end{align*}
\item
  \textbf{Identity:}

Let $e_i$ be the identity of $G_i$ for all $i$. Then $(e_1, \ldots, e_n)$ is an identity for $G_1 \times \ldots \times G_n$
\item
  \textbf{Inverses:}

The element $(g_1, \ldots , g_n)$ has the inverse $(g_1^{-1}, \ldots g_n^{-1})$.

\end{itemize}

\end{proof}
\emph{Example}: Consider the group $C_1 \times C_2$ which has elements $(1,1), (1,-1),(-1,1),(-1,-1)$

\begin{table}[]

\centering
\label{my-label}
\begin{tabular}{l|llll}
  & e & a & b & c \\ 
  \hline
e & e & a & b & c \\
a & a & e & c & b \\
b & b & c & e & a \\
c & c & b & a & e
\end{tabular}
\end{table}

Recall the group of symmetries of the rectangle, $\{I, R_\pi, T_x, T_y\}$.

This has group table:

\begin{table}[h]
\centering
\label{my-label}
\begin{tabular}{l|llll}
       & I      & $R_\pi$ & $T_x$   & $T_y$   \\
       \hline
I      & I      & $R_\pi$ & $T_x$   & $T_y$   \\
$R_\pi$ & $R_\pi$ & I      & $T_y$   & $T_X$    \\
$T_x$   & $T_x$   & $T_y$   & I      & $R_\pi$ \\
$T_\pi$ & $T_y$   & $T_x$   & $R_\pi$ & I     
\end{tabular}
\end{table}

So we have an isomorphism given by

$e \rightarrow I$\\
$a \rightarrow R\pi$\\
$b \rightarrow T_x$\\
$c \rightarrow T_y$\\

Recall from M1P2 that $S_4$ has a subgroup $$\{id,  (12)(34)=x, (13)(24)=y, (14)(23)=z\}.$$

This has group table 
\begin{table}[]
\centering
\begin{tabular}{l|llll}
   & id & x  & y  & z \\
   \hline
id & id & x  & y  & z \\
x  & x  & id & z  & y \\
y  & y  & z  & id & x \\
z  & z  & y  & x  & id 
\end{tabular}
\end{table}

So this too is isomorphic to $C_2 \times C_2$, via $e \rightarrow id, a \rightarrow id, a \rightarrow x, b \rightarrow y, c \rightarrow z$.

\emph{Example}: $C_2 \times C_2 \times C_2$, which has elements $(\pm 1, \pm 1, \pm 1)$. We see there are 8 elements and $x^2=(1,1,1)=e$ for any $x$.

\begin{proposition}
\begin{itemize}
\item $$|G_1 \times \ldots \times G_n| =|G_1||G_2|\ldots|G_n|.$$
\item If \emph{all} of the groups $G_i$ are abelian, then so is $G_1 \times \ldots \times G_n$. But if \emph{any} of $G_1, \ldots G_n$ is not abelian then netiher is $G_1 \times \ldots \times G_n$.
\item Let $(g_1, \ldots g_n) \in G_1 \times \ldots \times G_n$. Then $\ord (g_1, \ldots , g_n) = lcm (\ord g, \ldots , \ord g_n)$.
Proof not known.
\end{itemize}
\end{proposition}

\section{Small groups}
\begin{enumerate}
  \item 
  Every group of order 1 is isomorphic to $\{1\}$.
\item Every group of prime order is cyclic, so isomorphic to $C_8$. So only one group of order $2,3,5,7,\ldots$

\item Groups of order 4: Certainly we have $C_4$. Any other group must have 3 elements of order 2, say $a,b,c$. Start our multiplication table:

%      e a b c
%  e | e a b c
%  a | a e
%  b | b    e
%  c | c     e

The six vacant places are all determined by the fact that every element must appear onee in every rw and evry column.

Complete table:
%      e a b c
%  e | e a b c
%  a | a e c b
%  b | b c e a
%  c | c b a e

So $G \equiv C_2 \times C_2$ (So every gorup of order 4 is abelian).

\item Groups of order 6.

  Certainly we have $C_6$. Any other group has 5 elements with orders 2 or 3, with at least one of order 2 (by prop. ?)

  \underline{Result:} if a group $G$ has only elements of order $\leq 2$, then $G$ is abelian. ( if $g^2 = e\,\forall g$, look at $(ab)^2 = e$ So $ab$\underline{ab}$=e$ and also $ab$\underline{ba}$=aa=e$. y left cancellation, $ab=ba$, for all $a,b$

  The abelian gorups of order 5 are $C_6, C_2 \times C_3$, which are isomorphic. If follows that every group of order 6 hs an element of order 3.

  Let $a$ be an element of order 3.

  Let $b$ be an element of order 2.

  \begin{equation}
    G = \langle a \rangle \cup b \langle a \rangle \text{ so } G = \left\{ e,a,a^2,b,ba,ba^2 \right\}
    \label{}
  \end{equation}

  Start multiplication table:

%  .   |e   a   a^2   b   ba   ba^2
%  e   |e   a   a^2
%  a   |a a^2   e
%  a   |2 a^2   e    a^2
%  b   |b ba    ba^2 e
%  ba  |ba ba^2 b
%  ba^2| ba^2 b ba

Note that if $ab=ba$ then we can check that $ab$ must have order 6, 
since $(ab)^2=a^2b^2=a^2$, $(ab)^3=a^3b^3=b.$ But $G$ is not cyclic, by assumption, so $ab\neq ba$.
Hence we must have $ab = ba^2$.

This allows us to complete the multiplication table:
\begin{equation}
  \begin{matrix}
     & e & a & a^2 & b & ba & ba^2 \\
   e & e & a & a^2 & b & ba & ba^2 \\
   a & a & a^2 & e & ba^2 & b & ba \\
   a^2 & a^2 & e & a & ba & ba^2 & b \\
   b & b & ba & ba^2 & e & a & a^2               \\
   ba & ba & ba^2 & b & a^2 & e & a            \\
   ba^2 & ba^2 & b & ba & a & a^2 & e
 \end{matrix}
\end{equation}

We can check that  $G$ is isomotphic to $S_3$. So the gorupsof order 6 are $C_6 and S_3$.

\item Groups of order 8.

  Abelian groups: $C_8,C_4\times C_2, C_2 \times C_2 \times C_2.$ 

  Non abelian group $D_8$.

  There is one other non abelian group.
  \begin{equation}
    Q_8 = \left\{ \pm1, \pm i, \pm j, \pm k \right\} \text{ where } i^2=j^2=k^2=-1.
    \label{}
  \end{equation}
  \begin{equation}
    \begin{matrix}
      ij=k & jk=i & ki=j \\
      ji=-k & kj=-i & ik=-j
    \end{matrix}
    \label{}
  \end{equation}
  This is the \emph{quaternion group}. 
  $Q_8$ has six elements of order 4. 
  $D_8$ only has two elements of order 4, so they are not isomorphic.

  Alternatively, lets define matrices

  \begin{equation*}
    A = \left( 
\begin{matrix}
  0 & 1 \\
  -1 & 0
\end{matrix} 
    \right),
   B = \left( 
   \begin{matrix}
     i & 0 \\
     0 & -i
   \end{matrix}
   \right),
   C = 
   \left( 
\begin{matrix}
  0 & -i \\
  -i & 0
\end{matrix}
   \right)
    \label{}
\end{equation*}

  So $A,B,C \in GL_2(\mathbb(C))$. I cleain that if $G = \left\{ \pm I, \pm A, \pm B, \pm C \right\}$

  Then $G \leq GL_2(\mathbb(C))$ and there is an isomorphism $G \equiv Q_8$ given by 

  \begin{equation*}
    \begin{matrix}
      f(\pm I ) = \pm 1 & f(\pm A) = \pm i \\
      f(\pm B ) = \pm j & f(\pm C ) = \pm k
    \end{matrix}
    \label{}
  \end{equation*}
  It follows that since matrix multiplication is associative, the multiplication on $Q_8$ is associative too!

\item All grooups of order 9 are abelian, so either $C_9$ or $C_3 \times C_3$.
  The groups of order 10 are $C_{10},D_{10}$ Groups of order 12 are $C_{12}, C_{6} \times C_2$ (the abelian ones), $D_{12},A_{4}$, and one other. 

  Groups of order 14 are $C_{14}$ and $D_{14}$. 

  Only gorups of order 15 is $C_{15}$.

  Therre are 14 groups of order 16.
\end{enumerate}

Helpful result:
\begin{proposition}
  Let $G$ be a group of even order. Then $G$ has an element of order 2.
\end{proposition}

\begin{proof}
  Let $S$ to be the subset of $G$ such that 
  \begin{equation*}
    S = \left\{ g \in G : \ord g > 2 \right\}
    \label{}
  \end{equation*}
  $g \in S \iff g^{-1} \in S$ (Since $\ord g^{-1} = \ord g$). And $g \neq g^{-1}$, since $g^2 \neq e$. So $S$ contains pairs $(g,g^{-1})$ of elements, so $|g|$ is even. 

  But $|G|=|S| + |\left\{ g : \ord g=2 \right\}| + 1$ (identitity). 

  So $|\left\{ g : \ord g = 2 \right\}|=|G|-|S|-1$ which is odd. 

  So $\left\{ g : \ord g = 2 \right\} \neq \emptyset.$

\end{proof}


\section{Homomorphisms}

For an isomorphism we had two conditions that a map $G \rightarrow H$ has to satisfy:

\begin{enumerate}
  \item bijection
  \item respects multiplication
\end{enumerate}

If we drop (2), we get the definition of a homomorphism.

\begin{definition}
  A map $f:G \rightarrow H$ is a \emph{homomorphism} if $f(g_1 g_2)=f(g_1)f(g_2)$ for all $g_1,g_2 \in G.$
\end{definition}

\underline{Examples}
\begin{enumerate}
  \item Any isomorphism is a homomorphism. In fact an isomorphism is a bijection homomorphism.
  \item Let $G,H$ be any groups. Then
    $f : G \rightarrow H$ defined by $f(g) = e_H$ is a homomorphism. (Check $f(g_1 g_2)=e_h, f(g_1)f(g_2)=e_He_h = e_h$). This is the ``trivial homomorphism'' $G \rightarrow H.$
  \item Let $G \leq H$. Then $f:G\rightarrow H$ defined by $f(g)=g$ is a homomorphism. (The inclusion map of $G$ into $H$.)
  \item Define $\text{sign}:S_n \rightarrow C_2$ by $\text{sign}(g)=\text{sign}(g)=+1 \text{ if } g $ is even, $-1$ if $g$ is odd. (We showed before that $\sgn(gh)=\sgn(g)\sgn(h)$.)
  \item Define $f:D_{2n} \rightarrow C_2$ by 

    $f(g)=1$ if $g$ a rotation, $-1$ if $g$ a reflection.

    This is a homomorphism. (We know rot $\times$ rot $=$ rot, ref $\times$ rot $=$ ref, rot $\times$ ref $=$ ref, ref $\times$ ref $=$ rot).

    So its easy to see that $f$ respects multiplication.

  \item Define $f:GL_2(F) \rightarrow F^\times$ by

    $F(g) = \text{det}g$. Then $f$ is a homomorphism. 
    (We know that $\text{det}(gh)=(\text{det}g)(\text{det}h)$ for $g,h \in GL_2(F)$.) This can be done for $n>2$ as well.
  \item Let $V,W$ be vector spaces, and let $T:V \rightarrow W$ be a linear transformation. Then $V$ and $W$ are groups under $+$ and $T$ is a homomorphism (We know that $T(v_1 + v_2)=T(v_1)+T(v_2)$ since $T$ preserves addition.) 
  \item Define $f: (\mathbb{R},+)\rightarrow (\mathbb{C}^\times,\times)$ by $f(x)=e^{2 \pi i x}.$ Then $f$ is a homomorphism. (Check: $f(x+y)=e^{2\pi i (x+y)}=e^{2\pi i x}e^{2 \pi i y}=f(x)f(y)$).
\end{enumerate}

\begin{proposition}
  Let $f: G \rightarrow H$ be a homomorphism. 
  \begin{enumerate}
    \item $f(e_G)=e_H$
    \item $f(g^{-1})=f(g)^{-1}$ for all $g \in G$.
    \item $\text{ord}f(g)$ divides $\text{ord}g$ for all $g \in G$.
    \item
  \end{enumerate}
\end{proposition}

\begin{proof}
  (1) and (2) are the same as for the corresponding parts of Proposition (3?).

  (3) \emph{Claim:} $f(g^k)=f(g)^k$ for all $k\in \mathbb{Z}.$
  \begin{proof}
    If $k \geq 0$, use induction. If $k<0$, show instead that $f\left( \left( g^{-1} \right)^{-k} \right)=f(g^{-1})^{-k}$. (i.e. replace $g$ with $g^{-1}$ and $k$ with $-k$). now suppose that $k=\text{ord}g$. Then

    $g^k = e_G$. $f(g^k)=f(e_G)=e_H$. Now by the claim, $f(g)^k=e_H$ and so we have $\text{ord}f(g)$ divides $k$.
  \end{proof}
\end{proof}

\begin{definition}
  Let $f:G \rightarrow H$ be a homomorphism. We define: 
  
  The \emph{image} of $f$: $\text{Im } f = \left\{ f(g) : g \in G \right\}$.

  The \emph{kernel} of $f$: $\text{Ker } f=\left\{ g \in G : f(g) = e_H \right\}$ 
\end{definition}

\begin{theorem}
  \begin{enumerate}
    \item 
  $\text{Im } f$ is a subgroup of $H$.
    \item
  $\text{Ker } f$ is a subgroup of $G$.
  \end{enumerate}
  \label{}
\end{theorem}

\begin{proof}
  \begin{enumerate}
    \item Check the axioms.

      $f(e_g)=e_H$ so $e_H \in \text{Im }f$.
      Suppose $h_1, h_2 \in \text{Im } f$. Then there exist $g_1, g_2 \in G$ such that $f(g_1)=h1, f(g_2)=h_2$. Now, $f(g_1 g_2)=f(g_1)f(g_2)=h_1 h_2$. So $\text{Im }f$ is closed under the group operation.

      Suppose $h \in \text{Im }f. $ Then $h=f(g)$ for some $g \in G$. Now $f(g^{-1})=f(g)^{-1}=h^{-1}$. So $\text{Im }f$ is closed under inverses. So $\text{Im } f \in H $.

    \item Again, check the subgroup conditions.
      $f(e_G)=e_H$, so $e_G \in \text{Ker }f$. Suppose $g_1, g_2 \in \text{Ker }f$. Then $f(g_1)=f(g_2)=e_H$. ow $f(g_1 g_2)=f(g_1)f(g_2)=e_He_H=e_h$. So $g_1 g_2 \in \text{ker }f$. Finally, let $g \in \text{Ker }f$. Then $f(g)=e_H$. now $f(g^{-1})=f(g)^{-1}=e_H^{-1}=e_H$, so $g^{-1} \in \text{ker }f$. So $\text{ker }f \leq G$.
  \end{enumerate}
\end{proof}

\begin{exmp}
  Suppose $G=S_3$. Let $f$ be a homomorphism from $G$ to some group $H$. What are the possibilties or $\text{Im }f$ and $\text{Ker }f$?

  Recell that for $g \in G$ we have $f(g)$ divided $\ord g$.

  In $S_3$ we have elements $(123)$, $(12)$ or orders 3 and 2 respectively. So
  \begin{enumerate}
    \item 
  $f\left( (123) \right)$ has order 3 or else $f\left( (123) \right)=e_H$ 
    \item 
  $f\left( (12) \right)$ has order 2 or else $f\left( (12) \right)=e_H$ 
  \end{enumerate}

  \textbf{Case 1:} $f\left( (123) \right)$ has order 3 and $f\left(  (12)\right)$ has order 2. In this case $|\text{Im }f|$ is divisible by both 2 and 3. But $|\text{Im }f| \leq 6$, so we must have $\text{Im } f=6$. So (ocnsidered as a map $G \rightarrow \Ima f$) $f$ is bijective. Hence $\Ima f \cong S_3$.

  \textbf{Case 2:} $f\left( (12) \right)=e_H$. In this case observe that $(123)=(13)(12)$.

  So $f\left( (123) \right)=f\left( (13) \right)e_H=f\left( (13) \right)$.

  But $f\left( (123) \right)$ divides 3 and $f\left( (13) \right)$ divides 2. Hence these must both be $e_H$. Now we see that $\ker f$ is a subgroup of $S_3$ containing $id, (123), (12), (13)$, so $\ker f=S_3$. And so $\Ima f = \left\{ e_H \right\}$. $f$ is the trivial homomorphism.

  \textbf{Case 3:} $f\left( (12) \right)$ has order 2, $f\left( (123) \right)=e_H$. Notice that $(13)(12)=(123)$, so $f\left( (13) \right)f\left( (12) \right)=e_H$. So $f\left( (13) \right)=f\left( (12) \right)^{-1}=f\left( (12) \right)$ (Since $f\left( (12) \right)$ has order 2).

  Similarly $(12)(23)=(123)$, so $f\left( (12) \right)=f\left( (23) \right)$. Hence the image is $\left\{ e_H,f\left( (12) \right) \right\}$, a group of order 2, and $\ker f=\left\{ e,(123),(132) \right\}$.

  If we take $\Ima f$ to be $C_2$, then $f$ is the sign homomorphism.

  \textbf{Observation:} Not every subgroup of $S_3$ occurred as the kernel of a homomorphism.
\end{exmp}

The next section explores which subgroups can occur.

Suppose $\func{f}{G}{H}$ is a homomorphism, and suppse $g \in \ker f$. Let $g_1$ be every element of $G$. Then I claim $g_1,g_0,g_1^{-1}\in \ker f$.

Check: 
\begin{align*}
  f(g_1\,g_0\,g_1^{-1}) &= f(g_1)f(g_0)f(g_1)^{-1}\\
  &= f(g_1)e_Hf(g_1)^{-1} \\
  &= e_H.
\end{align*}

So if $K=\ker f$, then $K=gKg^{-1}=\left\{ gKg^{-1} : k \in K \right\}$ for all $g \in G$

\begin{proposition}
  Let $G$ be a group and let $H$ be a subgroup of $G$. The following are equivalent:
  \begin{enumerate}
    \item $H=gHg^{-1}$ for all $g \in G$.
    \item $Hg=gH$ for all $g \in G$.
    \item Every left coset is a right coset
    \item Every right coset is a left coset
    \item $gHg^{-1} \subseteq H$ for all $g \in G$
  \end{enumerate}
\end{proposition}

\begin{proof}
  $(2) \implies (3)$: Suppose that $Hg=gH$ for all $g$. Then if we take a left coset $gH$ then this is equal to the right coset $Hg$.

  $(3)\implies(2):$ Suppose that every left coset of $H$ is a right coset. So for $g \in G$, $gH=Hx$ for some $x \in G$. But $g \in gH$, and so $g \in Hx$. Hence $Hx=Hg$, so $gH=Hg$.

  $(2)\implies(4):$ The same as above.

  $(1)\implies(2):$ 
  Consider $x \in gHg^{-1}$. So $x=ghg^{-1}$ for some $h \in H$. Now $xg=gh$. So $(gHg^{-1})g=\left\{ (ghg^{-1}) g \right\}=\left\{ gh : h \in H \right\}. $ $gHg^{-1}=H$, so $(gHg^{-1})g=Hg$ Hence $Hg = gH$.

  $(2)\implies(1):$ Suppose that $Hg=gH$. Then $Hgg^{-1}=gHg^{-1} \implies H=gHg^{-1}$.

  $(1) \implies (5):$ Since $gHg^{-1}=H \implies gHg^{-1} \subseteq H$.

  $(5) \implies (1):$ Supppse that $gHg^{-1}\subseteq H$ for all $g \in G$. Fix an element $g$. Then $g^{-1} \in G$ and so $g^{-1}H\left( g^{-1} \right)^{-1} \subseteq H$, so $g^{-1}Hg \subseteq H$. So $g\left( g^{-1}Hg \right)g^{-1} \subseteq gHg^{-1}$, and it follows that $H \subseteq gHg^{-1}$. But we know that $gHg^{-1} \subseteq H$, and so $H=gHg^{-2}$.
\end{proof}

\begin{definition}
  A subgroup $H$ of a group $G$ with the property that $gHg^{-1}=H$ for all $g \in G$ (or any of the other equivalent properties from Prop. 16) is said to be a \emph{normal} subgroup of $G$. We write $H \trianglelefteq G$ to mean that it is a normal subgroup.
\end{definition}

\begin{exmp}
  \begin{enumerate}
    \item Every group $G$ has normal subgroups $\left\{ e \right\}$ and $G$. (Easy to see that $g\left\{ e \right\}g^{-1}=\left\{ geg^{-1} \right\}=\left\{ gg^{-1} \right\}=\left\{ e \right\}).$

      And for $g \in G$, $gG = G=Gg$, so $gGg^{-1}=G$ by condition $(2)$ from Prop. 16.

    \item If $G$ is abelian then every subgroup of $G$ is normal. 

      (Let $H \leq G$, and let $g \in G$. Then $gH=\left\{ gh:h \in H \right\}=\left\{ hg:h \in H \right\}=Hg$).

    \item Let $\nsg{A_n}{S_n}$. Let $g \in S_n$, and let $h \in A_n$. Then $\sgn \left( ghg^{-1} \right)=\sgn(g)\sgn(h)\sgn(g)^{-1}=\sgn(h)=+1$.

      So $gHg^{-1} \in A_n$ for all $h \in A_n$. Hence $gA_ng^{-1} \subseteq A_n$ for all $g$ and this condition $(5)$ from Prop. 16.

      \begin{enumerate}
        \item More generally, if $|G:H|=2$ (index 2 -  so $G = H \cup Hg$ for some $g \in G$) then $\nsg{H}{G}$. We see that the left cosets are $H$ and $gH$, so every left coset is a right coset.
      \end{enumerate}

    \item Let $G = D_{2n}$, and let $R$ be the rotation subgroup. Then $|G:R|=2$, so $\nsg{R}{G}$. In fact every subgroup of $R$ is normal in $G$.

      $R$ is cyclic, say $R = \langle r \rangle$.
      A subgroup of $R$ is $\gene{r^k}$ for some $k$. Let $g \in D_{2n}$. If $g$ is a rotation then $g\gene{r^k}=g\left\{ r^{kl}:l \in \mathbb{Z} \right\}=\left\{ gr^{kl}:l \in \mathbb{Z} \right\}= \left\{ r^{kl}:l \in \mathbb{Z} \right\}=\gene{r^k}g$, since any two rotations in $D_{2n}$ commute.

      If $g$ is a reflection then $$g\gene{r^k}=\left\{ gr^{kl}:l \in \mathbb{Z} \right\}=\left\{ r^{-kl}g:l \in \mathbb{Z} \right\}=\gene{r^k}g,$$ since $ba=a^{-1}b$ for any rotation $a$ and reflection $b$ in $D_{2n}$. So in either case the left and right cosets are the same.

    \item $G \times H$ has a subgroup  $G \times \left\{ e_H \right\}$ isomorphic to $G$. Call this $\hat{G}$. Then $\nsg{\hat{G}}{G \times H}$. 
\begin{align*}
  (g,h)\hat{G}(g^{-1},h^{-1}) &= g^{-1}Gg^{-1} \times h\left\{ e_H \right\}h^{-1} \\
  &= G \times \left\{ e_H \right\} = \hat{G}
\end{align*}
Since $\nsg{G}{G}$, and $\nsg{\left\{ e_H \right\}}{H}$.

\item $V_4 = \left\{ id,(12)(34), (13)(24),(14)(23) \right\} \trianglelefteq S_4$. (Check that the right and left cosets are equal.)
  \end{enumerate}
\end{exmp}

\subsection*{Non examples}
\begin{enumerate}
  \item $\gene{(12)}=\left\{ id,(12) \right\}$ in $S_3$ is not normal.

    $(13)\left\{ id, (12) \right\}=\left\{ (13),(123) \right\} \neq \left\{ id,(12) \right\}(13) = \left\{ (13)(132) \right\}$.

  \item if $t$ is a reflection in $D_{2n}$, then $\gene{t}=\left\{ id,t \right\}$ is not normal in $D_{2n}$. (Assume $n \geq 3$)

    Take $g$ a rotation of maximal order. Then 
    $$g\gene{t}=\left\{ g,gt \right\}=\left\{g,tg^{-1}\right\}\neq \left\{ g,tg \right\}=\gene{t}g$$
\end{enumerate}

\begin{proposition}
  Let $\func{f}{G}{H}$ be a group homomorphism. Then $\ker f \trianglelefteq G$. 
\end{proposition}

\begin{proof}
  Let $K = \ker f$, and let $k \in K$. For $g \in G$, we have 
  \begin{align*}
  f(ghg^{-1})=f(g)f(h)f(g)^{-1} &= f(g)e_Hf(g)^{-1} \,, k \in \ker f\\ &= f(g)f(g)^{-1} = e_H
  \end{align*}

  So $gkg^{-1} \in K$ for all $g \in K$, $k \in K$, so $gKg^{-1} \subseteq K$ for all $g \in G$. This is condition (5) from Prop. 16, so $K \trianglelefteq G$.
\end{proof}
\end{document}
